% set counter to n-1:
\setcounter{chapter}{0}

\chapter{Introduction}

Visualization of a potential outcome is a crucial aspect of any surgical procedure, especially if the procedure is cosmetic in nature. Through the use of 3D models, surgeons can
provide their patients with a visual representation of the potential outcome of a surgical procedure. This thesis is done in collaboration with Arbrea Labs, a company that specializes in the development of
software solutions for plastic surgeons. Arbrea Labs has developed a pipeline that allows surgeons to create 3D models of their patients' breasts. The dataset of 3D models used in this thesis is
provided by Arbrea Labs.

The main goal of this thesis is to develop a cross-platform application that allows users to modify the appearance of the breasts and visualize the potential outcomes of a surgical procedure. We want to implement
a way to change the patient's breast with the use of simple UI elements like sliders, but also with a more sophisticated method, which allows the user to change the breast by drawing a line where he wants the breast to be.

The aim is to establish a unified codebase capable of operating on various platforms, with a particular focus on the web and Android.

This thesis is structured in the following chapters:
\begin{itemize}
    \item Chapter \ref{ch02-related-work} gives an overview of the background and some related work.
    \item Chapter \ref{ch03_methods} discusses in-depth all the methods used in this thesis.
    \item Chapter \ref{ch04_evaluation} gives an experimental evaluation and compares the methods used in this thesis.
    \item Chapter \ref{ch05-conclusion-and-outlook} summarizes the results and gives an outlook on future work, solving some of the problems that occurred during the development of this thesis.
\end{itemize}

